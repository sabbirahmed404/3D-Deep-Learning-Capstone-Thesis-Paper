Thank you for using the Official FCCU Computer Science Undergraduate Thesis template! This template was originally authored by Benjamin Williams of University of Lincoln. If you have any troubles, requests, or issues please get in touch with \fbox{\centering\texttt{\textbf{mumtazsheikh@fccollege.edu.pk}}}.

\section{Cautionary Note}
It is worth noting that:

\begin{enumerate}
    \item This template \emph{should} provide \emph{all} of the required formatting for your undergraduate thesis, as required by the module;
    \item Provides auxillary environments for typesetting ease -- such as ludography and in-built features of \LaTeX, and;
    \item Formats references automatically in accordance with Harvard referencing, a requirement for the FCCU's DCS undergraduate dissertations.
\end{enumerate}


This is your introductory chapter. The template is pre-formatted in accordance to the Department of Computer Science's undergraduate thesis requirements. For example, this is double-spaced with the correct margin values and header formats. 

\section{Template Structure}
Over on the left of this screen (assuming you're in Overleaf), you should see the file structure of this template. There are a few folders and files which are important. In this template, the \texttt{chapters} folder is where all the chapters throughout the document are located. This includes other sections, such as the abstract, acknowledgements and appendices.

By contrast, the \texttt{preamble} folder contains files which are used before the document is rendered. Think of this like the \texttt{head} tag in HTML: the files in this folder provide important meta-data (such as your name, student number etc) prior to the document being rendered. Your first task should be to modify this template, by opening up \texttt{preamble/details.tex} and inserting your own details. If you wish to use a package in your document, you can easily add it to the \texttt{preamble/packages.tex} file, and it will be imported.  Furthermore, the \texttt{preamble/bib-setup.tex} file is used to import references and set-up Bib\LaTeX, but that is covered later in this document.


\section{Table of Tables (ToT) and Table of Figures (ToF)}
At the beginning of this document, you probably noticed these two sections which were included before this chapter, and after the abstract. These are the table of figures and the table of tables. Lets include an image, and notice how it appears in the table of figures above as Figure \ref{fig:tof-example}.

\begin{figure}[htb]
    \centering
    \includegraphics[width=3cm, height=3cm]{logo.pdf}
    \caption{FCCU's logo, used in the title page of this thesis.}
    \label{fig:tof-example}
\end{figure}

In addition, here is an example of a table. Whilst not strictly required, it is worth noting that Table \ref{tbl:example-table} (on the next page) is formatted in accordance to normal usage in scientific articles. If you wish to create tables, it's probably best to use a \LaTeX~tables generator, like \url{http://tablesgenerator.com/}.

\begin{table}[htb]
\centering
\caption{A list of house plants and their details.}
\begin{tabular}{l|l|l|l}
\textbf{Name} & \textbf{Type} & \textbf{Location}         & \textbf{Size} \\
\hline\hline
Billy         & Peace Lily    & Living room, on bookshelf & Massive!      \\
Jezza         & Calathea      & Living room, on fireplace & Fairly big    \\
Juno          & Laurel Bush   & Kitchen Window            & Fairly big    \\
Jade          & Jade Plant    & Bathroom Window           & Mediumish     \\
Bernard II    & Ivy           & Kitchen Window            & Big           \\
Shelley       & Geranium      & Bedroom Window            & Small        \\
Benth         & Kalanchoe     & Desktop                   & Big
\end{tabular}
\label{tbl:example-table}
\end{table}

\section{Typesetting Math}
As computer scientists, we often need to describe things with mathematical symbols or operators. Luckily, mathematical typesetting is part and parcel of \LaTeX, with many solutions to this problem. This template uses the \texttt{amsmath} package for mathematical typesetting, alongside the \texttt{amssymb} package. For example, you can use in-line math formatting (with single dollars) like this:

\begin{framed}
For each $\mathbf{p} \in P$, the weight $w = \left( \mathbf{p}_0 \cdot \mathbf{p}_2 \right) - \mathbf{p}_1$ was calculated. This linear operation had a time complexity of $\mathcal{O}(n)$.
\end{framed}

Double dollars will centre your math without an equation number, like so:

\begin{framed}
$$ f(x) = \int_{-\infty}^{\infty}{\hat{f}(\xi)e^{2\pi i x \xi}d\xi} $$
\end{framed}

And finally the \texttt{align} environment will allow to reference and align equations, like in Equation \ref{eqn:example1} and \ref{eqn:example2} below. Notice how both $=$ symbols are aligned horizontally.

\begin{framed}
\begin{align}
a &= b + 1 
\label{eqn:example1} \\
\frac{\hbar^2}{2m}\nabla^2\Psi + V(\mathbf{r})\Psi
&= -i\hbar \frac{\partial\Psi}{\partial t}    
\label{eqn:example2}
\end{align}
\end{framed}

But what if you want to align two or more equations but have no equation numbers? Well, thats a job for the \texttt{align*} environment:

\begin{framed}
\begin{align*}
d(\mathbf{a}, \mathbf{b}) &= \sqrt{(a_0 - b_0)^2 + (a_1 - b_1)^2 + ... + (a_n - b_n)^2} \\
        &= \sqrt{\sum_{i=0}{n}{(a_i - b_i)^2}} \\
        &= || \mathbf{a} - \mathbf{b} || \\
        &= \sqrt{(\mathbf{a} - \mathbf{b}) \cdot (\mathbf{a} - \mathbf{b})}
\end{align*}
\end{framed}


\section{Hmm, the margins seem big? Especially on the left of the page?}
Yeah, that's okay. This is done by design as your thesis will be printed and bound on one side, so you need extra room on the left so the text doesn't fall into the bind on your page.

The margins used in this template are the standards for postgraduate research theses, which are also applicable for undergraduate theses too. The reason for the large margins and big spacing is so those marking your work can write in notes and red-pen your thesis easily. So yeah, don't worry about it!

\cleardoublepage